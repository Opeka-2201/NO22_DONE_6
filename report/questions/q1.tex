\section{Model}

\subsection{Implementation of the model in linear programming}
\label{sec:model}
In order to to compute the winning distribution law of the RCVS, one can implement a linear formulation of the model.
The linear formulation is based on the following variables:
\begin{itemize}
  \item $A$ the voting matrix where $A_{i,j}$ represents the results of a duel between the $i$-th and $j$-th candidates.
  The elements of the matrix are computed following this rule : for each voter, if the $i$-th canditate is ranked higher than the $j$-th candidate the element $A_{i,j}$ is incremented
  and the element $A_{j,i}$ is decremented.
  \item $p$ the probability vector where $p_i$ represents the probability that the $i$-th candidate wins.
  \item $e$ the vector of ones.
\end{itemize}

The linear formulation of the model is the following:
\begin{align*}
  &\min_{p} \sum p^T A\\
  \text{s.t. } &p^T e = 1\\
  &p \geq 0\\
  &p^T A \geq 0  
\end{align*}

This formulation was implemented in the file \verb|q1_model.jl| using the \verb|JuMP| and \verb|Gurobi| packages.

\subsection{Application of the RCVS to an example}
We want to apply the Condorcet winning system to the following example where edges goes from loser to winner with relatives weights :

\begin{figure}[!h]
  \centering
  \includegraphics[width=0.5\textwidth]{figs/graph.png}
  \caption{Example of voting graph}
  \label{fig:q1_example}
\end{figure}

In this example, there will no be any Condorcet winner because the graph is not a directed acyclic graph, indeed there is a cycle between the candidates $B$, $C$ and $D$.
The RCVS is then needed to solve this problem. We simply need to compute the $A$ matrix respecting the graph represented in figure \ref{fig:q1_example} following the rule described in section \ref{sec:model}:

$$A = \begin{pmatrix}
  0 & -40 & 0 & 0\\
  40 & 0 & -60 & 80\\
  0 & 60 & 0 & -20\\
  0 & -80 & 20 & 0
\end{pmatrix}$$
\newpage
When lauching the \verb|q1_model.jl| file with this matrix as input, we obtain the following lottery for the RCVS :

\begin{table}[!h]
  \centering
  \begin{tabular}{|c|c|c|c|c|}
  \hline
  Candidate   & $A$   & $B$     & $C$   & $D$     \\ \hline
  Probability & $0.0$ & $0.125$ & $0.5$ & $0.375$ \\ \hline
  \end{tabular}
  \caption{Lottery probabilities for each candidate}
  \label{tab:q1_prob}
\end{table}

\subsection{Discussion of the dual variables and optimal dual basis}


\subsection{Solution of the linear in a linear system}
Assuming that there exists a RCVS winning lottery can be reformulated as saying that there won't be any tie between two candidates, an assumption that is probaliticly true, as when the number of voters gets bigger, the probability of a tie gets smaller.

\subsection{\textit{Bonus} : Comparaison of the RCVS with an alternative voting system}