\section{Model}

\subsection{Implementation of the model in linear programming}
In order to to compute the winning distribution law of the RCVS, one can implement a linear formulation of the model.
The linear formulation is based on the following variables:
\begin{itemize}
  \item $A$ the voting matrix where $A_{i,j}$ represents the results of a duel between the $i$-th and $j$-th candidates.
  The elements of the matrix are computed following this rule : for each voter, if the $i$-th canditate is ranked higher than the $j$-th candidate the element $A_{i,j}$ is incremented
  and the element $A_{j,i}$ is decremented.
  \item $p$ the probability vector where $p_i$ represents the probability that the $i$-th candidate wins.
\end{itemize}

The linear formulation of the model is the following:
\begin{align*}
  &\min_{p} \sum p^T A\\
  \text{s.t. } &p^T 1 = 1\\
  &p \geq 0\\
  &p^T A \geq 0  
\end{align*}

\subsection{Application of the RCVS to an example}


\subsection{Discussion of the dual variables and optimal dual basis}


\subsection{Solution of the linear in a linear system}


\subsection{\textit{Bonus} : Comparaison of the RCVS with an alternative voting system}